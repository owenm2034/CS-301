\documentclass{article}

\begin{document}

\title{CS 301 Assignment 4\\[0.5cm]\large Student ID: 200482797}
\author{Owen Monus}
\date{April 11th, 2024}

\maketitle

\begin{enumerate}
	\item
	      \begin{enumerate}
		      \item You are given an one-address machine with an accumulator and the following memory
		            values:
		            \newline Word 20 contains 40.
		            \newline Word 30 contains 50.
		            \newline Word 40 contains 60.
		            \newline Word 50 contains 70
		            \newline
		            \newline
		            What values do the following instructions load into the accumulator?
		            \begin{enumerate}
			            \item LOAD IMMEDIATE 20
			                  \newline This loads 20 into the accumulator

			            \item LOAD DIRECT 20
			                  \newline This loads 40 into the accumulator

			            \item LOAD INDIRECT 20
			                  \newline This loads 60 into the accumulator

			            \item LOAD IMMEDIATE 30
			                  \newline This loads 30 into the accumulator

			            \item LOAD DIRECT 30
			                  \newline This loads 50 into the accumulator

		            \end{enumerate}

		      \item An address field in an instruction contains a decimal value 14. Where is the corresponding
		            operand located for\dots
		            \begin{enumerate}
			            \item Immediate addressing?
			                  \newline The operand is in the instruction

			            \item Direct addressing?
			                  \newline The operand is in address 14

			            \item Indirect addressing?
			                  \newline The operand is pointed to by the address contained in address 14

			            \item Register addressing?
			                  \newline The operand is in register 14

			            \item Register indirect addressing?
			                  \newline The operand is pointed to by the address contained in register 14
		            \end{enumerate}
	      \end{enumerate}

	\item Let the address stored in the program counter (PC) be designated by the symbol X1. The
	      instruction stored in X1 has an address part (operand reference) X2. The operand needed to execute the
	      instruction is stored in the memory word with address X3. An index register contains the value X4. What
	      is the relationship between these various quantities if the addressing mode of the instruction is (a)
	      direct; (b) indirect; (c) PC relative; (d) indexed?

			(a) X3 is equal to X2.
		  	\newline (b) X3 is the value pointed to by the address stored in X2. 
			\newline (c) X3 is X1 + X2 + 1, since the PC is incremented before execution. 
			\newline (d) X3 is X2 + X4, the index register.

	\item
	      \begin{enumerate}
		      \item A PC-relative mode branch instruction is 3 bytes long. The address of the instruction, in
		            decimal, is 256028. Determine the branch target address if the signed displacement in the instruction is
		            -31.
				\paragraph{} PC = \(256031\), since instructions are 3 bytes long, the resultant address is: \(256,000\).
		      \item A PC-relative mode branch instruction is stored in memory at address \(620_{10}\). The branch is
		            made to location \(530_{10}\). The address field in the instruction is 10 bits long. What is the binary value in the
		            instruction? Note that (x)10 indicates that x is a decimal number.
					
					\paragraph{} The address field would contain the value \(1110100101_2\), or \(-91_{10}\)
	      \end{enumerate}

	\item
	      \begin{enumerate}
		      \item If the last operation performed on a computer with an 8-bit word was an addition in
		            which the two operands were 00000010 and 00000011, what would be the value of the following flags?
		            \begin{enumerate}
			            \item Overflow = 0
			            \item Sign = 0
		            \end{enumerate}

		      \item If the last operation performed on a computer with an 8-bit word was a subtraction in
		            which the two operands were 11110000 and 0010100 (i.e., 11110000 - 00010100 was performed), what
		            would be the value of the following flags?
		            \begin{enumerate}
			            \item Carry = 1
			            \item Zero = 0
			            \item Overflow = 0
		            \end{enumerate}

		      \item A pipelined processor has a clock rate of 2.5 GHz and executes a program with 1.5 million
		            instructions. The pipeline has five stages, and instructions are issued at a rate of one per clock cycle.
		            Ignore penalties due to branch instructions and out-of-sequence executions. What is the speedup of
		            this processor for this program compared to a nonpipelined processor? You can assume that (a) each
		            instruction goes through all five stages of the pipeline; (b) the five stages are of equal duration; (c) all of
		            the stages can be performed in parallel; and (d) there are no conditional branch instructions.
					
					\centering
					\( \frac{1,500,000 \times 5}{1,500,000 + 5 - 1} \approx 4.999...\)
	      \end{enumerate}

	\raggedright
	\item Write a summary of what you have learned in this course in about 1000 words.
	      \paragraph{} A digital system consists of two notions: architecture and organization.
	      Architecture refers to the attributes of the system visible to a programmer. These include: instruction set, data types, memory
	      addressing mode, and I/O mechanisms. Organization refers to the operational units and their
	      interconnections. Organizational attributes involve control signals, interfaces, etc.

	      \paragraph{} We describe a computer as a hierarchical system. Structure refers to the way in which the
	      components are interrelated. We describe structure for a single processor computer and a multicore computer.
	      Function refers to the operation of each individual component as part of the structure. There
	      are four basic functions that a computer can perform: data processing, data storage, data movement,
	      and control.

	      \paragraph{} To improve speed, some techniques are in the following. Pipelining
	      enables a processor to work simultaneously on multiple instructions by performing a different stage for
	      each of the multiple instructions at the same time. Branch prediction: The processor predicts which instructions are likely to be
	      processed next, and it pre-fetches the instructions so that the processor is kept busy.
	      Superscalar execution: the ability to issue more than one instruction in every processor
	      clock cycle. Data flow analysis: The processor analyzes which instructions are dependent on each other’s
	      results, to create an optimized schedule of instructions.
	      Speculative execution: Using branch prediction and data flow analysis, some processors
	      speculatively execute instructions ahead of their actual appearance in the program. This enables the
	      processor to keep its execution engines as busy as possible.

	      \paragraph{} The use of multiple processors on a same chip is called multicore.
	      Multicore provides the potential to increase performance without increasing the clock rate.
	      Cache memory can be shared among processors in multicore. Cache memory can also be placed in the
	      same chip. When cache memory is larger, there can be multiple levels of cache.
	      Within a processor, the increase in performance is roughly proportional to the square root of the
	      increase in complexity.

	      \paragraph{} The basic function of a computer is to execute programs, which is achieved by interconnecting CPU,
	      memory, and I/O components. Interconnection can be bus interconnection or point-to-point
	      interconnection. The processing required for a single instruction is called an instruction cycle. Two steps of an instruction
	      cycle are the fetch cycle and the execute cycle. At the beginning of each instruction cycle, the processor fetches an instruction from memory. In a
	      typical processor, a register called the program counter (PC) holds the address of the instruction to be
	      fetched next. Unless told otherwise, the processor always increments the PC after each instruction fetch
	      so that it will fetch the next instruction in sequence. The fetched instruction is loaded into a register in the processor known as the instruction register (IR).
	      The instruction contains bits that specify the action the processor is to take. The processor interprets the
	      instruction and performs the required action. In general, these actions fall into four categories.
	      Processor-memory (load and store), Processor-I/O Data processing. Control.

		  \paragraph{} Temporal locality refers to a program's tendency to access memory units in the near future that were 
		  recently accessed in the past, such as in iteration loops or when using constants and temporary variables. Spatial 
		  locality refers to a program's tendency to access memory units whose addresses are close to each other. This 
		  occurs when a unit of memory is accessed, like when processing arrays. There are 4 types of memory access: Sequential 
		  access involves accessing records in a linear order, typical of tape units. Direct access allows accessing individual 
		  blocks or records with unique addresses based on physical location, often associated with disk units. Random access 
		  enables selecting and accessing any location in memory directly through unique addresses, as seen in main memory and 
		  some cache memory. Associative access, a type of random access memory, retrieves words based on their content rather 
		  than their address. Each location has its addressing mechanism, with constant retrieval time. Cache memories may use 
		  associative access.

		  \paragraph{} Greater memory capacity typically comes with slower access time, while higher access speeds usually result in 
		  greater cost per bit. Similarly, larger capacities tend to have lower cost per bit, while faster access times are often 
		  associated with higher costs per bit.
		  
		  \paragraph{} Instruction types are categorized as follows: data processing instructions encompass arithmetic and logic 
		  operations; data storage instructions involve moving data between registers and memory locations; data movement 
		  instructions handle input/output operations; and control instructions include branching operations.

		  \paragraph{} Immediate addressing directly embeds the operand value within the instruction, saving a memory cycle but 
		  restricting the operand size to the address field. Direct addressing uses the address field to denote the location of 
		  the operand in memory. Indirect addressing employs a word in memory to hold the full-length address of the operand, 
		  expanding the address space but necessitating two memory references for operand retrieval. Register addressing replaces 
		  memory addresses with register references, facilitating faster execution but offering limited address space. Register 
		  indirect addressing, akin to indirect addressing, substitutes memory addresses with a register reference, sharing similar 
		  advantages and limitations. Displacement addressing merges direct and register indirect addressing, utilizing two address 
		  fields to calculate the effective address based on a register's contents.
		  
		  \paragraph{} There are three types of registers: user-facing, control, and status. User-visible registers enable assembly 
		  language programmer to minimize main memory references by optimizing use of registers. Control registers are used by the
		  control unit to control the operation of the processor and by operating system to control the execution of programs.
		  These registers are typically not visible to the user. The status register(s) contain condition codes plus other status
		  information.

		  \paragraph{} This is only \textbf{some} of the content covered in CS 301.

\end{enumerate}

\end{document}