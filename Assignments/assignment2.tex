\documentclass{article}
\begin{document}

\title{CS 301 Assignment 2\\[0.5cm]\large Student ID: 200482797}
\author{Owen Monus}
\date{March 4th, 2024}

\maketitle

\begin{enumerate}
    \item Consider two different machines, with two different instruction sets, both of which have a
    clock rate of 200 MHz. The following measurements are recorded on the two machines running a given
    set of programs. Determine the effective CPI, MIPS rate, and execution time for each machine.
        
    \begin{table}[htbp]
        \centering
        \renewcommand{\arraystretch}{2}
        \caption{Equations for Machines \(A\) and \(B\)}
        \label{tab:equations}
        \begin{tabular}{cc}
            \hline
            Machine \(A\) & Machine \(B\) \\
            \hline
            \( CPI = \frac{\sum_{i=1}^{n}{CPI_i \times I_i}}{{n}} \) & 
            \( CPI = \frac{\sum_{i=1}^{n}{CPI_i \times I_i}}{{n}} \) \\
            \( CPI = \frac{8 \times 1 + 4 \times 3 + 2 \times 4 + 4 \times 3}{18} \) & 
            \( CPI = \frac{10 \times 1 + 8 \times 2 + 2 \times 4 + 4 \times 3}{24} \) \\
            \( CPI = \frac{40}{18} \approx 2.22 \) & \( CPI = \frac{46}{24} \approx 1.92\) \\
            \hline
            \( MIPS = \frac{f \times 10^6}{CPI \times 10^6} \) & 
            \( MIPS = \frac{f \times 10^6}{CPI \times 10^6} \) \\
            \( MIPS = \frac{200 \times 10^6}{2.22 \times 10^6} \) & 
            \( MIPS = \frac{200 \times 10^6}{1.92 \times 10^6} \) \\ %come back and fix these calcs
            \hline
            \( T = CPI \times \frac{1}{f} \times I_c \) & 
            \( T = CPI \times \frac{1}{f} \times I_c \) \\
            \( T = 2.22 \times \frac{1}{200 \times 10^6} \times 18,000,000 \) & 
            \( T = 1.92 \times \frac{1}{200 \times 10^6} \times 24,000,000 \) \\
            % \( T = 1998000 \) & 
            \( T = 0.1998\) \\
        \end{tabular}
    \end{table}

    \pagebreak
    \item Consider the following code:
    \begin{verbatim}
for (i = 0; i < 20; i++)
    for (j = 0; j < 10; j++)
        a[i] = a[i] * j
    \end{verbatim}
    \begin{enumerate}
        \item Give one example of the spatial locality in the code. 
        \newline An example of spatial locality in this code can be seen in the array $a$. 
        The data in $a$ is stored linearly and accessed sequentially.
        \item Give one example of the temporal locality in the code.
        \newline An example of temporal locality in this code can be seen in the line that modifies $a$.
        This line of code is repetitively accessed in the for loops, lending temporal locality to the instruction
        and the data accessed in the array $a$
    \end{enumerate}

    \item What do you mean by CPI, average CPI, and MIPS? Derive the formula for MIPS, in terms
    of clock frequency and average CPI.
    \newline 
    
\end{enumerate}




\end{document}