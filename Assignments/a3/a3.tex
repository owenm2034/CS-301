\documentclass{article}

\begin{document}

\title{CS 301 Assignment 3\\[0.5cm]\large Student ID: 200482797}
\author{Owen Monus}
\date{March 16th, 2024}

\maketitle

\begin{enumerate}
    \item
    \begin{enumerate}
        \item Represent the following decimal numbers in twos complement using 16 bits: +512, -29.
        \newline
        512 → 0000001000000000
        \newline
        -29 → 1111111111100011
        \item Represent the following twos complement values in decimal:
        \newline 
        1101011 → -21
        \newline
        0101101 → 45
    \end{enumerate}
    
    \item
    \begin{enumerate}
        \item The $r$'s complement of an $n$-digit number $N$ in base $r$ is defined as $r^n - N$ for $N \neq 0$ and $0$ for $N = 0$. Find the tens complement of the decimal number $13,250$.
        \paragraph{} 13,250's 10's complement is 86,750. This was obtained by taking the 9's complement and adding one.
        
        \item Calculate (72530-13250) using tens complement arithmetic. Assume that the rules are
        similar to those for twos complement arithmetic.
        \item Take 10s comp of 13250, subtract that from 72530. This gives 159280. Discard leading 1, gives 59280
    \end{enumerate}

    \pagebreak
    \item Use the Booth's algorithm to multiply 23 (multiplicand) by 29 (multiplier), where each
    number is represented using 6 bits.
    \begin{verbatim}
// Booth's Algorithm with M = 23, N = 29
A = 0
B = 010111 // 23
C = 011101 // 29
C_-1 = 0
n = 6

while(n > 0) {
    if (LSB(C) == 0 && C_-1 == 1)
        A = A + B
    if (LSB(C) == 1 && C_-1 == 0)
        A = A - B
    
    n = n - 1
    A||C||C_-1 = (A||C||C_-1) <<< 1 
}
    \end{verbatim}
    \pagebreak

    \item
    \begin{table}[htbp]
        \centering
        % \renewcommand{\arraystretch}{2}
        \caption{Booth's Algorithm: \(23 \times 29\)}
        \begin{tabular}{c|c|c|c|c}
            n & A & C & C\(_{-1}\) & B \\
            \hline
            6\(_{10}\) & 000000 & 011101 & 000000 & 010111 \\
            5\(_{10}\) & 000000 & 111010 & 000000 & 101110 \\
            4\(_{10}\) & 000000 & 111010 & 000000 & 101110 \\
            % Step 2 & ... & ... & ... \\
        \end{tabular}
    \end{table}
    
    \item 

\end{enumerate}

\end{document}