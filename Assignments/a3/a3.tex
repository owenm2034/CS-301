\documentclass{article}

\begin{document}

\title{CS 301 Assignment 3\\[0.5cm]\large Student ID: 200482797}
\author{Owen Monus}
\date{March 16th, 2024}

\maketitle

\begin{enumerate}
    \item
    \begin{enumerate}
        \item Represent the following decimal numbers in twos complement using 16 bits: +512, -29.
        \newline
        512 → 0000001000000000
        \newline
        -29 → 1111111111100011
        \item Represent the following twos complement values in decimal:
        \newline 
        1101011 → -21
        \newline
        0101101 → 45
    \end{enumerate}
    
    \item
    \begin{enumerate}
        \item The $r$'s complement of an $n$-digit number $N$ in base $r$ is defined as $r^n - N$ for $N \neq 0$ and $0$ for $N = 0$. Find the tens complement of the decimal number $13,250$.
        \paragraph{} 13,250's 10's complement is 86,750.
        
        \item Calculate \(72,530 - 13,250\) using tens complement arithmetic. Assume that the rules are
        similar to those for twos complement arithmetic.
        \paragraph{} Take 10s comp of 13250, add that to 72530.

        \(72530+86750 = 159,280\)
    \end{enumerate}

    \pagebreak
    \item Use the Booth's algorithm to multiply 23 (multiplicand) by 29 (multiplier), where each
    number is represented using 6 bits.
    \begin{verbatim}
// Booth's Algorithm with M = 23, N = 29
A = 0
B = 010111 // 23
C = 011101 // 29
C_-1 = 0
n = 6

while(n > 0) {
    if (LSB(C) == 0 && C_-1 == 1)
        A = A + B
    if (LSB(C) == 1 && C_-1 == 0)
        A = A - B
    
    n = n - 1
    A||C||C_-1 = (A||C||C_-1) >>> 1 
}
    \end{verbatim}
    \begin{table}[htbp]
        \centering
        % \renewcommand{\arraystretch}{2}
        \caption{Booth's Algorithm: \(23 \times 29\)}
        \begin{tabular}{c|c|c|c|c|c}
            \hline
            Operation & n & A & C & C\(_{-1}\) & B \\
            \hline
            initial value & 6\(_{10}\) & 000000 & 011101 & 0 & 010111 \\
            \(A = A - B\) & 6\(_{10}\) & 101001 & 011101 & 0 & 010111 \\
            right shift & 6\(_{10}\) & 110100 & 101110 & 1 & 010111 \\
            \(A = A + B \) & 5\(_{10}\) & 001011 & 101110 & 1 & 010111 \\
            right shift & 5\(_{10}\) & 000101 & 110111 & 0 & 010111 \\
            \(A = A - B\) & 4\(_{10}\) & 101110 & 110111 & 0 & 010111 \\
            right shift & 4\(_{10}\) & 110111 & 011011 & 1 & 010111 \\
            right shift & 3\(_{10}\) & 111011 & 101101 & 1 & 010111 \\
            right shift & 2\(_{10}\) & 111101 & 110110 & 1 & 010111 \\
            \(A = A + B\) & 1\(_{10}\) & 010100 & 110110 & 1 & 010111 \\
            right shift & 1\(_{10}\) & 001010 & 011011 & 0 & 010111 \\
            % Step 2 & ... & ... & ... \\
        \end{tabular}
        \newline
        \newline Result = 0b001010011011 \( = 667_{10}\)
    \end{table}

    \pagebreak
    \item 
    \begin{verbatim}
// One instruction machine
LOAD E
MUL F
STORE T
LOAD D
SUB T
STORE X
LOAD B
MUL C
ADD A
DIV X
STORE X

// Two instruction machine
MOVE X,F
MUL X,E
MOVE T,D
SUB T,X

MOVE X,C
MUL X,B
ADD X,A
DIV X,T

// Three instruction machine
MUL T,E,F
SUB T,D,T
MUL X,B,C
ADD X,X,A
DIV X,X,T
    \end{verbatim}

    \pagebreak
    \item Both an arithmetic left shift and a logical left shift
    correspond to a multiplication by 2 when there is no overflow. If overflow occurs, arithmetic and logical
    left shift operations produce different results, but the arithmetic left shift retains the sign of the
    number.
    \begin{table}[htbp]
        \centering
        \caption{Shifts for Decimal Numbers}
        \begin{tabular}{c|c|c|c|c}
        \hline
        $d$ & $2$'s & LLS & ALS & Condition \\
        \hline
        -16 & 10000 & 00000 & 10000 & overflow \\
        -15 & 10001 & 00001 & 10010 & overflow \\
        -14 & 10010 & 00100 & 10100 & overflow \\
        -13 & 10011 & 00011 & 10110 & overflow \\
        -12 & 10100 & 01000 & 11000 & overflow \\
        -11 & 10101 & 01010 & 11010 & overflow \\
        -10 & 10110 & 01100 & 11100 & overflow \\
        -9 & 10111 & 01110 & 11110 & overflow \\
        -8 & 11000 & 10000 & 10000 & \shortstack{no overflow} \\
        -7 & 11001 & 10010 & 10010 & \shortstack{no overflow} \\
        -6 & 11010 & 10100 & 10100 & \shortstack{no overflow} \\
        -5 & 11011 & 10110 & 10110 & \shortstack{no overflow} \\
        -4 & 11100 & 11000 & 11000 & \shortstack{no overflow} \\
        -3 & 11101 & 11010 & 11010 & \shortstack{no overflow} \\
        -2 & 11110 & 11100 & 11100 & \shortstack{no overflow} \\
        -1 & 11111 & 11110 & 11110 & \shortstack{no overflow} \\
        0 & 00000 & 00000 & 00000 & \shortstack{no overflow} \\
        1 & 00001 & 00010 & 00010 & \shortstack{no overflow} \\
        2 & 00010 & 00100 & 00100 & \shortstack{no overflow} \\
        3 & 00011 & 00110 & 00110 & \shortstack{no overflow} \\
        4 & 00100 & 01000 & 01000 & \shortstack{no overflow} \\
        5 & 00101 & 01010 & 01010 & \shortstack{no overflow} \\
        6 & 00110 & 01100 & 01100 & \shortstack{no overflow} \\
        7 & 00111 & 01110 & 01110 & \shortstack{no overflow} \\
        8 & 01000 & 10000 & 00000 & overflow \\
        9 & 01001 & 10010 & 00010 & overflow \\
        10 & 01010 & 10100 & 00100 & overflow \\
        11 & 01011 & 10110 & 00110 & overflow \\
        12 & 01100 & 11000 & 01000 & overflow \\
        13 & 01101 & 11010 & 01010 & overflow \\
        14 & 01110 & 11100 & 01100 & overflow \\
        15 & 01111 & 11110 & 01110 & overflow \\
        \end{tabular}
        \label{tab:shifts}
        \end{table}
        \paragraph{} As shown in the table, when an operation results in overflow, the Arithmetic Left
        Shift and Logical Left Shift result in different results. The Arithmetic Left Shift retains 
        the sign of the number, whereas the Logical Left Shift does not.
        

\end{enumerate}

\end{document}